\documentclass{article}


%% Bring the margins down to 1 inch, like the old ``fullpage'' package.
\usepackage[margin=1.0in]{geometry}

\usepackage{graphicx}
\usepackage{caption}
\usepackage{subcaption}


%% Gives the equivalent of one-and-a-half line spacing.
\linespread{1.3}

\begin{document}

\section{Overview of Shanes's original data}

Our earlier work determined that analyses at the phylum-level did not
provide as much predictive efficacy as did models built on
family-level and order-level taxa.  Therefore, we focused on these
taxa.  These taxa were obtained from 6 cadavers on 16 different days.
However, taxa counts are not available for
- subject A1 on days 7 and 9
- subject A4 on day 7


Since our last meeting, I realized that the data collected from
subject A3 on day 40 (degree day 1130) appears to be defective.  The
total number of taxa (including unclassifed taxa) counted for this
cadaver and day was 54.  This is several orders of magnitude less than
the counts for the other cadavers and days, which ranged from 18050 to
93665 (both orders and families).  For this reason, the taxa from
subject A3 on day 40 were omitted from the analyses.  The omission of
the A3, day 40 data does not affect the analyses which only consider
the first two weeks or data.

For the analyses considered all time steps, we have 92 observations
(16 days $\times$ 6 cadavers, minus the 4 missing cadaver-day
combinations mentioned in the previous paragraphs).  For the analyses
considering the first 15 days (approximately two weeks), we have 57
observations (10 days $\times$ 6 cadavers, minus the 3 missing
cadaver-day combinations occuring during this period).


\section{Cross-validation procedures}

I had some concerns about how well our model would perform in
prediction mode.  We have only 6 cadavers, and we only have
observations at certain time steps.  I'm continuing to use
cross-validation, but I've reduced the training set to 80\% of the
data, reserving 20\% to use as a test set.  When using all the data,
this means training on a set of 74 observations, and testing on a size
of 18.  Previously, the split was 90\% and 10\%.  This made very
little difference in the choice of models, which relieved some of my
concern.

Since our last meeting, I figured out how to parallelize some of the
computational work, which allowed me to use some additional
cross-valiation runs and to try additional models.


\section{Results for order-level taxa}

\subsection{All time steps}

Original cutoff considers 20 possible order-level taxa.
Stricter cutoff considers 15 possible order-level taxa.

\begin{tabular}{ll}
Inclusion cutoff & Units  & RMSE & Explained variation\\
1\% at least once  & Orig.~units & 230.19 & 82.81\%\\
1\% at least twice & Orig.~units & 230.65 & 82.74\%\\
%% 1\% at least once  & Sqrt.~units &   4.38 & 87.62\%\\
%% Didn't do the sqrt runs with the stricter cutoff.
\end{tabular}




\subsection{First 15 days}


\end{document}
