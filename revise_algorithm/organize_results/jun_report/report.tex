\documentclass{article}


%% Bring the margins down to 1 inch, like the old ``fullpage'' package.
\usepackage[margin=1.0in]{geometry}

\usepackage{graphicx}
\usepackage{caption}
\usepackage{subcaption}


%% Gives the equivalent of one-and-a-half line spacing.
\linespread{1.3}

\begin{document}

\section{Overview of Shanes's original data}

Our earlier work determined that analyses at the phylum-level did not
explain nearly as much of the variability as did models built on
family-level and order-level taxa.  So, we focused on the family and
order levels.  These taxa were obtained from 6 cadavers on 16 different
days.  However, taxa counts are not available for\\
- subject A1 on days 7 and 9\\
- subject A4 on day 7


Since our last meeting, I realized that the data collected from
subject A3 on day 40 (degree day 1130) appears to be defective.  The
total number of taxa (including unclassifed taxa) counted for this
cadaver and day was 54.  This is several orders of magnitude less than
the counts for the other cadavers and days, which ranged from 18050 to
93665 (both orders and families).  For this reason, the taxa from
subject A3 on day 40 were omitted from all the analyses.  The omission
of the A3, day 40 data does not affect the analyses which only
consider the first two weeks or data.

For the analyses which considered all time steps, we have 92
observations (16 days $\times$ 6 cadavers, minus the 4 missing
cadaver-day combinations mentioned in the previous paragraphs).  For
the analyses considering the first 15 days (approximately two weeks),
we have 57 observations (10 days $\times$ 6 cadavers, minus the 3
missing cadaver-day combinations occuring during this period).


\section{Cross-validation procedures}

I had some concerns about how well our model would perform in
prediction mode.  We have only 6 cadavers, and we only have
observations at certain time steps.  I'm continuing to use
cross-validation, but I've reduced the training set to 80\% of the
data, reserving 20\% to use as a test set.  (Previously, the split was
90\% and 10\%.)  When using all the data, this means training on a set
of 74 observations, and testing on a size of 18.  When using just the
first 15 days of data, this means training on a set of 46
observations, and testing on a size of 11.  For each model that I fit,
I did 1000 cross-validation runs.  For each of these runs, we randomly
select which 20\% of the observations will be left out of the random
forest algorithm and used to test the model.  The change from 90\% to
80\% made very little difference in the choice of models, which
relieved some of my concern.

Since our last meeting, I figured out how to parallelize some of the
computational work, which allowed me to use some additional
cross-valiation runs and to try additional models, compared to what I
had done previously.


\section{Results for order-level taxa}

\subsection{All time steps}

Original cutoff considers 20 possible order-level taxa.
Stricter cutoff considers 15 possible order-level taxa.

\begin{tabular}{llll}
Inclusion cutoff & Units  & RMSE & Explained variation\\
1\% at least once  & Orig.~units & 230.19 & 82.81\%\\
1\% at least twice & Orig.~units & 230.65 & 82.74\%\\
1\% at least once  & Sqrt.~units &   4.38 & 87.62\%\\
Didn't do the sqrt runs with the stricter cutoff.
\end{tabular}




\subsection{First 15 days}

Original cutoff considers 16 possible order-level taxa.
Stricter cutoff considers 14 possible order-level taxa.

\begin{tabular}{llll}
Inclusion cutoff & Units  & RMSE & Explained variation\\
1\% at least once  & Orig.~units & 53.93 & 87.59\%\\
1\% at least twice & Orig.~units & 54.68 & 87.25\%\\
1\% at least once  & Sqrt.~units & 2.01 & 90.81\%\\
1\% at least twice & Sqrt.~units & 2.02 & 90.68\%
\end{tabular}



\section{Results for family-level taxa}

\subsection{All time steps}

Original cutoff considers 49 possible family-level taxa.  When using
the original cutoff, the algorithm picks Listeriaceae as an important
predictor, even though this family only gets above the 1\% mark for
only one individual for only one day.
%% Top four predictors: 

Stricter cutoff considers 39 possible family-level taxa.
%% Top four predictors: Peptostreptococcaceae, Microbacteriaceae,
%% Turicibacteraceae, Oxalobacteraceae

\begin{tabular}{llll}
Inclusion cutoff & Units  & RMSE & Explained variation\\
1\% at least once  & Orig.~units & 198.58 & 87.21\%\\
1\% at least twice & Orig.~units & 207.18 & 86.08\% %%\\
%% 1\% at least once  & Sqrt.~units & 4.18 & 88.70\%\\
%% Didn't do the sqrt runs with the stricter cutoff.
\end{tabular}



\subsection{First 15 days}

Original cutoff considers 37 possible family-level taxa.
Stricter cutoff considers 35 possible family-level taxa.

\begin{tabular}{llll}
Inclusion cutoff & Units  & RMSE & Explained variation\\
1\% at least once  & Orig.~units & 64.98 & 81.99\%\\
1\% at least twice & Orig.~units & 64.25 & 82.39\%\\
1\% at least once  & Sqrt.~units & 2.31 & 87.87\%\\
1\% at least twice & Sqrt.~units & 2.28 & 88.20\%
\end{tabular}

\end{document}

